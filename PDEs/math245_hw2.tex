% Options for packages loaded elsewhere
\PassOptionsToPackage{unicode}{hyperref}
\PassOptionsToPackage{hyphens}{url}
%
\documentclass[
]{article}
\usepackage{amsmath,amssymb}
\usepackage{iftex}
\ifPDFTeX
  \usepackage[T1]{fontenc}
  \usepackage[utf8]{inputenc}
  \usepackage{textcomp} % provide euro and other symbols
\else % if luatex or xetex
  \usepackage{unicode-math} % this also loads fontspec
  \defaultfontfeatures{Scale=MatchLowercase}
  \defaultfontfeatures[\rmfamily]{Ligatures=TeX,Scale=1}
\fi
\usepackage{lmodern}
\ifPDFTeX\else
  % xetex/luatex font selection
\fi
% Use upquote if available, for straight quotes in verbatim environments
\IfFileExists{upquote.sty}{\usepackage{upquote}}{}
\IfFileExists{microtype.sty}{% use microtype if available
  \usepackage[]{microtype}
  \UseMicrotypeSet[protrusion]{basicmath} % disable protrusion for tt fonts
}{}
\makeatletter
\@ifundefined{KOMAClassName}{% if non-KOMA class
  \IfFileExists{parskip.sty}{%
    \usepackage{parskip}
  }{% else
    \setlength{\parindent}{0pt}
    \setlength{\parskip}{6pt plus 2pt minus 1pt}}
}{% if KOMA class
  \KOMAoptions{parskip=half}}
\makeatother
\usepackage{xcolor}
\usepackage[margin=1in]{geometry}
\usepackage{graphicx}
\makeatletter
\def\maxwidth{\ifdim\Gin@nat@width>\linewidth\linewidth\else\Gin@nat@width\fi}
\def\maxheight{\ifdim\Gin@nat@height>\textheight\textheight\else\Gin@nat@height\fi}
\makeatother
% Scale images if necessary, so that they will not overflow the page
% margins by default, and it is still possible to overwrite the defaults
% using explicit options in \includegraphics[width, height, ...]{}
\setkeys{Gin}{width=\maxwidth,height=\maxheight,keepaspectratio}
% Set default figure placement to htbp
\makeatletter
\def\fps@figure{htbp}
\makeatother
\setlength{\emergencystretch}{3em} % prevent overfull lines
\providecommand{\tightlist}{%
  \setlength{\itemsep}{0pt}\setlength{\parskip}{0pt}}
\setcounter{secnumdepth}{-\maxdimen} % remove section numbering
\usepackage{derivative}
\usepackage{tcolorbox}
\usepackage{xcolor}
\ifLuaTeX
  \usepackage{selnolig}  % disable illegal ligatures
\fi
\IfFileExists{bookmark.sty}{\usepackage{bookmark}}{\usepackage{hyperref}}
\IfFileExists{xurl.sty}{\usepackage{xurl}}{} % add URL line breaks if available
\urlstyle{same}
\hypersetup{
  pdftitle={MATH 245 Homework 2},
  pdfauthor={Ruby Krasnow and Tommy Thach},
  hidelinks,
  pdfcreator={LaTeX via pandoc}}

\title{MATH 245 Homework 2}
\author{Ruby Krasnow and Tommy Thach}
\date{2024-02-15}

\begin{document}
\maketitle

\hypertarget{done-feel-ok-1-2-5-7}{%
\subsection{Done, feel OK: 1, 2, 5, 7}\label{done-feel-ok-1-2-5-7}}

\hypertarget{done-very-low-confidence-4}{%
\subsection{Done, very low confidence:
4}\label{done-very-low-confidence-4}}

\hypertarget{redo-6}{%
\subsection{Redo: 6}\label{redo-6}}

\hypertarget{not-started-3-8}{%
\subsection{Not started: 3, 8}\label{not-started-3-8}}

\hypertarget{question-1}{%
\subsection{Question 1}\label{question-1}}

Determine the region in which the given equation is hyperbolic,
parabolic, elliptic, or singular.

\begin{enumerate}
\def\labelenumi{\alph{enumi})}
\tightlist
\item
  \[u_{xx}+y^2u_{yy}+u_x-u+x^2=0 \]
\end{enumerate}

\(a=1, b=0, c=-y^2\), so we have \(b^2-ac=0-(-y^2)=y^2\). This will be
positive everywhere except for \(y=0\), so the equation is hyperbolic
where \(y \neq 0\) and parabolic for \(y=0\).

\begin{enumerate}
\def\labelenumi{\alph{enumi})}
\setcounter{enumi}{1}
\tightlist
\item
  \[u_{xx}-yu_{yy}+xu_x+yu_y+u=0 \]
\end{enumerate}

\(a=1, b=0, c=-y\), so we have \(b^2-ac=0-(-y)=y\). Thus, the equation
will be hyperbolic where \(y>0\), parabolic where \(y=0\), and elliptic
where \(y<0\).

\hypertarget{question-2}{%
\subsection{Question 2}\label{question-2}}

Using a factorization similar to the wave equation, solve the following
IVP: \begin{equation}
    \begin{cases}
      u_{xx}+2u_{xy}-3u_{yy}=0 & x\subset \mathbb{R},\; y> 0\\
      u(0,x)=\sin{x} & x\subset \mathbb{R}\\
      u_y(0,x) = x & x\subset \mathbb{R}
    \end{cases}       
\end{equation}

First, we can factor the equation as follows:
\[\left( \pdv{}{x}+3\pdv{}{y}\right) \left(\pdv{}{x}-\pdv{}{y}\right)u=0\]
or
\[\left( \partial_{x}+3\partial_{y}\right) \left(\partial_{x}-\partial_{y}\right)u=0\]

Then set \(\left(\pdv{}{x}-\pdv{}{y}\right)u=v\), giving us

\(\left( \pdv{}{x}+3\pdv{}{y}\right)v= v_x+3v_t=0\)

which we know has the solution \(v(x,y)=f(3x-y)\), so
\[u_x-u_y=f(3x-y)\] Now we can reorder our equation as
\[\left(\partial_{x}-\partial_{y}\right)
\left( \partial_{x}+3\partial_{y}\right)u=0\] and set
\(w=\left( \partial_{x}+3\partial_{y}\right)u\)

Then \[w_x-w_y=0 \]

Which we know has the solution \(w(x,y)=g(x+y)\). So
\(u_x+3u_y=g(x+y)\), which gives us a system of two equations:

\begin{equation}
\begin{cases}
u_x-u_y=f(3x-y)
u_x+3u_y=g(x+y)
\end{cases}
\end{equation}

Subtract the first equation from the second: \[4u_y=-f(3x-y)+g(x+y) \]

Now we can integrate with respect to \(y\) to find that:
\[u(x,y)=F(3x-y)+G(x+y)\] where \(F\) is the antiderivative of \(-f\)
with respect to \(y\) and \(G\) is the antiderivative of \(g\) with
respect to \(y\).

Using the fact that \(u(0,x)=\sin{x}\),

\begin{equation} \label{eq:1} u(0,x)=\sin{x}=F(3x)+G(x) \end{equation}
now replace \(x\) with a new neutral variable, \(\alpha\) and
differentiate: \[\sin{\alpha}=F(3\alpha)+G(\alpha) \]
\begin{equation} \label{eq:2} \cos{\alpha}=3F'(3\alpha)+G'(\alpha) \end{equation}

But we can also differentiate \(u(x,y)=F(3x-y)+G(x+y)\) with respect to
\(y\) to get \[u_y(x,y)=-F'(3x-y)+G'(x+y)\] but from our initial
conditions, we know \[u_y(0,x)=-F'(3x-0)+G'(x+0)=x\]

Let's replace \(x\) by our neutral variable \(\alpha\) and solve for
\(F'\): \[F'(\alpha)=G'(3\alpha)-\alpha\] Now plug this into
(\ref{eq:2}): \[\cos{\alpha}=3G'(\alpha)-3\alpha+G'(\alpha)\]

\[G(\alpha)= \frac{1}{4} \int {\cos{\alpha}+3\alpha} = \frac{\sin{\alpha}}{4}+ \frac{3 \alpha^2}{8} \]
So that means (\ref{eq:1}) becomes:
\[\sin{\alpha}=F(3\alpha)+\frac{\sin{\alpha}}{4}+ \frac{3 \alpha^2}{8}\]

\[F(\alpha)=\frac{3\sin{(\frac{\alpha}{3})}}{4}- \frac{\alpha^2}{24}\]

Which means \(u(x,y)=F(3x-y)+G(x+y)\) becomes

\[u(x,y)=\frac{3}{4}\sin{\left(x-\frac{y}{3}\right)}- \frac{(3x-y)^2}{24}+  \frac{\sin{(x+y)}}{4}+ \frac{3(x+y)^2}{8}=\]

\begin{tcolorbox}[colback=white, title=Solution]
$$u(x,y)=\frac{3}{4}\sin{\left(x-\frac{y}{3}\right)}+\frac{\sin{(x+y)}}{4}+ xy+  \frac{y^2}{3} $$
\end{tcolorbox}

\begin{tcolorbox}[colback=blue!5!white,colframe=blue!75!black,title=Check solution]

 $$u_y=\frac{-1}{4}\cos{\left(x-\frac{y}{3}\right)}+\frac{\cos{(x+y)}}{4}+ x+  \frac{2y}{3} $$
  $$u_{yy}=\frac{-1}{12}\sin{\left(x-\frac{y}{3}\right)}-\frac{\sin{(x+y)}}{4}+ \frac{2}{3} $$
 
$$u_x=\frac{3}{4}\cos{\left(x-\frac{y}{3}\right)}+\frac{\cos{(x+y)}}{4}+y$$
 
$$u_{xx}=\frac{-3}{4} \sin{\left( x-\frac{y}{3} \right)} - \frac{\sin{(x+y)}}{4}$$
  
  
$$u_{xy}=\frac{1}{4}\sin{\left(x-\frac{y}{3}\right)}-\frac{\sin{(x+y)}}{4}+1$$


Check that $u_{xx}+2u_{xy}-3u_{yy}=0$
$$
\left(\frac{-3}{4}+ \frac{2}{4}+\frac{1}{4}\right)\sin{\left(x-\frac{y}{3}\right)}+
\left(\frac{-1}{4}+ \frac{-2}{4}+\frac{3}{4}\right)\sin{(x+y)}+(0+2-3)=0
$$

\end{tcolorbox}

\hypertarget{question-3}{%
\subsection{Question 3}\label{question-3}}

Solve the Neumann boundary value problem for the wave equation on half
line: \begin{equation}
    \begin{cases}
      u_{tt}=c^2u_{xx}+f(t,x) & 0<x<\infty\\
      u(0,x)=\phi{x} & 0<x<\infty\\
      u_t(0,x) = \psi{x} & 0<x<\infty\\
      u_x(t,0) = h(t) & t>0
    \end{cases}       
\end{equation}

\hypertarget{question-4}{%
\subsection{Question 4}\label{question-4}}

Consider the 3D wave equation for \(u(t,x,y,z)\):

\[u_{tt}=c^2\Delta u \quad\quad (x,y,z) \in \mathbb{R}^3, \quad t>0\]

Change the coordinates to spherical coordinates. Assume the solution is
spherically symmetric, so that \(u(t,x,y,z)=u(t,r)\) and does not depend
on \(\theta\) and \(\phi\). Find the solution for \(u(0,r)=0\) and

\begin{equation} u_t(0,r) = \begin{cases}
  1  & \lvert{r}\rvert \leq 1 \\
  0 &  \lvert{r}\rvert > 1
\end{cases}
\end{equation}

Hint: use the substitution \(u(t,r)= \frac{1}{r} w(t,r)\).

First, we need to derive the formula for the Laplacian in spherical
coordinates.

We know the equation for the Laplacian in polar coordinates is:

\[\frac{\partial^2}{\partial r^2}+\frac{1}{r}\frac{\partial}{\partial r}+ \frac{1}{r^2}\frac{\partial^2}{\partial \theta^2}\]

Now let's convert to spherical coordinates:

\[r=\sqrt{x^2+y^2+z^2}=\sqrt{s^2+z^2}\]

\[x=s \cos{\phi}, \quad y=s \sin{\phi}, \quad z=r \cos{\theta}\]
\[s=r \sin{\theta}\]

By the two-dimensional Laplacian, we have

\[u_{zz}+u_{ss}=u_{rr}+\frac{1}{r}u_r+\frac{1}{r^2}u_{\theta\theta},\]

\[u_{xx}+u_{yy}=u_{ss}+\frac{1}{s}u_s+\frac{1}{s^2}u_{\phi\phi}\] We add
these two equations and cancel \(u_s\) to get

\[\Delta u = u_{rr}+\frac{1}{r}u_r+\frac{1}{r^2}u_{\theta\theta}+\frac{1}{s}u_s+\frac{1}{s^2}u_{\phi\phi}\]

Now since \(u\) doesn't depend on \(\theta\) or \(\phi\), we have
\[\Delta u = u_{rr}+\frac{1}{r}u_r+\frac{1}{s}u_s=u_{rr}+\frac{1}{r}u_r+\frac{1}{r \sin{\theta}}u_s\]

\[ \frac{\partial u}{\partial s} = \frac{\partial u}{\partial r} \frac{\partial r}{\partial s}+\frac{\partial u}{\partial\theta}\frac{\partial\theta}{\partial s}+\frac{\partial u}
{\partial\phi}\frac{\partial\phi}{\partial s}=u_r\frac{1}{ \sin{\theta}}+0+0=u_r\frac{s}{ r}\]

So with our change of variables, we have

\[u_{tt}=c^2 \left(u_{rr} + \frac{2}{r}u_r \right)\]

Now set \(w=ru\), or \(u= \frac{w}{r}\). Then
\[w_t=ru_t, \quad w_{tt}=ru_{tt}, \quad u_{tt}= \frac{w_{tt}}{r} \]
\[w_t=ru_t, \quad w_{tt}=ru_{tt}, \quad u_{tt}= \frac{w_{tt}}{r} \]
\[u_r = \frac{w_r}{r}-\frac{w}{r^2} \]

\[u_{rr} = \frac{w_{rr}}{r}-\frac{2w_r}{r^2}+\frac{2w}{r^3} \] So
\(u_{tt}=c^2 \left(u_{rr} + \frac{2}{r}u_r \right)\) becomes

\[\frac{w_{tt}}{r} =c^2 \left(\frac{w_{rr}}{r}-\frac{2w_r}{r^2}+\frac{2w}{r^3} + \frac{2}{r} \left(\frac{w_r}{r}-\frac{w}{r^2} \right) \right),\]
which simplifies to \[w_{tt}=c^2 w_{rr}.\] But this is just the wave
equation, and we can use d'Alembert's formula to find the solution:
\[w(t,r)= \frac{\varphi(r+ct)+\varphi(r-ct)}{2}+\frac{1}{2c} \int_{r-ct}^{r+ct}{\psi(s)ds}\]
Since \(\varphi=0\),
\[w(t,r)= \frac{1}{2c} \int_{r-ct}^{r+ct}{\psi(s)ds}\]

Now we have 4 cases:

Case 1: \(r-ct \geq -1, r+ct \leq 1\)
\[w(t,r)= \frac{1}{2c} \int_{r-ct}^{r+ct}{s\;ds}\]

\[= \frac{1}{4c} ((r+ct)^2-(r-ct)^2)\]
\[= \frac{1}{4c}(r^2+2crt+c^2t^2-r^2+2crt-c^2t^2)= \frac{4crt}{4c}=rt  \]

Case 2:\(r-ct < -1, r+ct>1\)
\[w(t,r)= \frac{1}{2c} \int_{-1}^{1}{s\;ds}=\]
\[w(t,r)= \frac{1}{4c} (1-1)=0\]

Case 3:\(r-ct < -1, r+ct \leq 1\)
\[w(t,r)= \frac{1}{2c} \int_{-1}^{r+ct}{s\;ds}\]

\[= \frac{1}{4c} ((r+ct)^2-1)\]

Case 4:\(r-ct \geq -1, r+ct>1\)
\[w(t,r)= \frac{1}{2c} \int_{r-ct}^{1}{s\;ds}\]
\[= \frac{1}{4c} (1-(r-ct)^2)\]

Since \(u=\frac{w}{r}\), this means we have

\begin{tcolorbox}[colback=white, title=Solution]
\begin{equation}u(t,r)=
    \begin{cases} 
      t & \text{ if } r-ct \geq -1, r+ct \leq 1\\
      0 & \text{ if } r-ct < -1, r+ct > 1\\
      \frac{1}{4rc} ((r+ct)^2-1) & \text{ if } r-ct < -1, r+ct \leq 1\\
      \frac{1}{4rc} (1-(r-ct)^2) &\text{ if }  r-ct \geq -1, r+ct > 1\\
    \end{cases}       
\end{equation}
\end{tcolorbox}

\hypertarget{question-5}{%
\subsection{Question 5}\label{question-5}}

Consider the following Dirichlet boundary value problem:
\begin{equation} 
\begin{cases}
  u_{tt}+x(t,x)u_t=u_{xx} & 0<x<1 \\
  u(0,x)=\phi(x) & 0<x<1 \\
  u_t(0,x)=\psi(x) & 0<x<1 \\
  u(t,0) = u(t,1)=0 & t \geq 0 \\
\end{cases}
\end{equation}

Assume that \(\lvert a(t,x)\rvert \leq m\) for some constant \(m\) and
all \(0<x<1\) and \(t\geq 0\). Let
\[E(t)= \frac{1}{2} \int_{0}^{1}{ \left(u_t(t,x)^2+u_x(t,x)^2\right)}\:dx \]

\begin{enumerate}
\def\labelenumi{(\arabic{enumi})}
\tightlist
\item
  Show that \begin{equation}\label{eqn:E} 
  \frac{dE(t)}{dt} \leq 2mE(t)
  \end{equation} for \(t \geq 0\).
\end{enumerate}

First differentiate \(E(t)\):

\[ \frac{dE}{dt}= \frac{d}{dt} \left[\frac{1}{2} \int_{0}^{1}{ \left(u_t^2+u_x^2\right)}\:dx \right]\]
\begin{equation} \label{eqn:prob5a}
=\frac{1}{2} \int_{0}^{1}{ \frac{\partial}{\partial t} \left(u_t^2+u_x^2\right)} \:dx
\end{equation}

\begin{equation} \label{eqn:prob5b}
=\frac{1}{2} \int_{0}^{1}{ 2u_t u_{tt}+ 2u_x u_{xt}}\:dx
\end{equation}

\begin{equation}\label{eqn:prob5c}
=\int_{0}^{1}{ u_t u_{tt}}\: dx + \int_{0}^{1}{2u_x u_{xt}}\:dx =: I+J.
\end{equation}

The equality (\ref{eqn:prob5a}) follows from differentiation under the
integral sign, while (\ref{eqn:prob5b}) follows from the chain rule for
partial derivatives.

Consider the integral \(J\) in (\ref{eqn:prob5c}). By integrating by
parts, we can move one of the partials \(\frac{\partial}{\partial x}\)
to the other factor, at the cost of introducing a minus sign and a
boundary term. Hence \(J\) becomes

\begin{equation} \label{eqn:J}
J=\int_{0}^{1}{ u_x u_{xt}}\: dx =- \int_{0}^{1}{u_xx u_t}\:dx + 
\left[ u_x u_t \right]_{x=0}^{x=1}.
\end{equation}

The boundary term vanishes, since \(u(t,0) \equiv u(t,1) \equiv 0\) for
\(t>0\) implies that \(u_t\) is identically zero at \(x=0,1\). So,
substituting (\ref{eqn:J}) for \(J\), equation (\ref{eqn:prob5c})
becomes
\[ I+J=\int_{0}^{1}{u_t u_{tt}} \;dx - \int_{0}^{1}{u_{xx} u_t} \;dx\]
\[= \int_{0}^{1} {u_t (u_{tt}-u_{xx})} \;dx\]
\begin{equation} \label{eqn:J2}
=\int_{0}^{1}{ u_t(-au_t) }\:dx
\end{equation}

\begin{equation} \label{eqn:J3}
=\int_{0}^{1}{ (-a)u_t^2 }\: dx.
\end{equation}

Here, equality (\ref{eqn:J2}) just uses the PDE. Since \(u_t^2 \geq 0\)
and \(-a \leq \lvert -a \rvert \leq m\), we see that
\((-a)u_t^2 \leq mu_t^2\). Hence, the expression in (\ref{eqn:J3})
satisfies the inequality

\begin{equation} \label{eqn:IJ}
I+J= \int_{0}^{1}{ (-a)u_t^2 }\:dx \leq \int_{0}^{1}{ mu_t^2 }\:dx \leq m \int_{0}^{1}{ u_t^2 + u_x^2 }\:dx =2mE.
\end{equation}

Where we also used the fact that \(u_x^2 \geq 0\) means that
\(m\int_{0}^{1}{ u_t^2 }\:dx \leq m\int_{0}^{1}{ (u_t^2 + u_x^2) }\:dx\).

The desired inequality (\ref{eqn:E}) follows from (\ref{eqn:IJ}) and the
fact that \(\frac{dE}{dt}=I+J.\)

\begin{enumerate}
\def\labelenumi{(\arabic{enumi})}
\setcounter{enumi}{1}
\tightlist
\item
  Use part (1) and show that
  \(\frac{d}{dt} \left( e^{-2mE(t)} \right) \leq 0\) for all
  \(t \geq 0\). Hence, by integration from \([0,t]\), we get that
\end{enumerate}

\begin{equation}\label{eqn:leq}
E(t) \leq e^{2mt} E(0) \text{ for all } t \geq 0.
\end{equation}

By the product rule,
\[ \frac{d}{dt} \left(e^{-2mt}E\right)=-2me^{-2mt}E+e^{-2mt}\frac{dE}{dt} \]
\[=e^{-2mt}\left(\frac{dE}{dt}-2mE\right) \]
\[\leq e^{-2mt} \cdot 0=0. \]

\begin{enumerate}
\def\labelenumi{(\arabic{enumi})}
\setcounter{enumi}{2}
\tightlist
\item
  If \(\phi(x)=\psi(x)=0\) for all \(0<x<1\), what does this say about
  \(E(t)\) for \(t \geq 0\) and hence about \(u(t,x)\) for \(t \geq 0\)?
\end{enumerate}

Since \(u(0,x) \equiv \varphi(x) \equiv 0\) for \(0<x<1\), we see that
\(u_x\) is identically zero at time \(t=0\). Similarly,
\(u_t(0,x) \equiv \psi(x) \equiv 0\) for \(0<x<1\). Thus at \(t=0\),

\begin{equation}\label{eqn:E0}
E(0)= \frac{1}{2} \int_{0}^{1}{ \left(u_t(t,0)^2+u_x(t,0)^2\right)}\:dx =\frac{1}{2} \int_{0}^{1}{0^2+0^2 \:dx=0.}
\end{equation}

But (\ref{eqn:E0}) together with the inequality (\ref{eqn:leq}) implies
that

\[0 \leq E(t)  \leq e^{2mt}E(0) \leq 0\] for all \(t \geq 0\). Hence the
energy \(E\) is identically zero. Since the integrand \(u_t^2+u_x^2\) is
nonnegative, this is only possible if \(u_t \equiv u_x \equiv 0\) for
\(t>0, 0<x<1\), meaning that \(u\) varies with neither time nor
position. But this implies that \(u\) must be identically zero
everywhere.

\begin{enumerate}
\def\labelenumi{(\arabic{enumi})}
\setcounter{enumi}{3}
\tightlist
\item
  Use the previous part to prove uniqueness of the following problem:
  \begin{equation} 
  \begin{cases}
    u_{tt}+a(t,x)u_t=u_{xx} & 0<x<1, t>0 \\
    u(0,x)=\phi(x) & 0<x<1 \\
    u_t(0,x)=\psi(x) & 0<x<1 \\
    u(t,0) = f(t) & t \geq 0 \\
    u(t,1) = g(t) & t \geq 0 \\
  \end{cases}
  \end{equation}
\end{enumerate}

Let \(u\) and \(v\) be two solutions, and define \(w:=u-v\). Observe
that \(w\) satisfies the original boundary value problem along with the
conditions specified in part (3), implying that \(w \equiv 0\). Hence
\(u-v\equiv 0\) and so any solution \(u\) must be unique. \(\square\)

\hypertarget{problem-6}{%
\subsection{Problem 6}\label{problem-6}}

Does the D'Alembert method work for the wave equation
\(u_{tt} = c(x)^2u_{xx}\)? What about \(u_{tt} = c(t)^2u_{xx}\)? Why?

Let's try the factorization method if \(c=c(x)\):
\[\left( \partial_{t}+c(x)\partial_{x}\right) \left(\partial_{t}-c(x)\partial_{x}\right)u=0\]
Set \(\xi=x+c(x)t\) and \(\eta=x-c(x)t\).

By the chain rule,

\[\frac{\partial}{\partial x} = \frac{\partial}{\partial \xi} \frac{\partial \xi}{\partial x}+\frac{\partial}{\partial\eta}\frac{\partial\eta}{\partial x}\]

\[=\partial_{\xi}(1+c')+\partial_{\eta}\,(1-c')\]

\[\frac{\partial}{\partial t} = \frac{\partial}{\partial \xi} \frac{\partial \xi}{\partial t}+\frac{\partial }{\partial\eta}\frac{\partial\eta}{\partial t}\]
\[=c\partial_{\xi}-c\partial_{\eta}\] When we plug these back into
\[\left( \partial_{t}+c\partial_{x}\right) \left(\partial_{t}-c\partial_{x}\right)u=0\]
we get

\[ \left(c\partial_{\xi}-c\partial_{\eta}+c\partial_{\xi}(1+c')+c\partial_{\eta}\,(1-c')\right)
\left(c\partial_{\xi}-c\partial_{\eta} -c\partial_{\xi}(1+c')-c\partial_{\eta}\,(1-c')\right)u=0\]

When \(c\) is constant, \(c'=0\) and this simplifies to
\(-4c^2 u_{\xi\eta}=0\), allowing us to integrate to find
\(u(x,t)=F(x+ct)+G(x-ct)\). But we cannot do the same simplification
when we have the \(c'\) terms, meaning we cannot solve the wave equation
by the same method when the wave speed is not constant.

Similarly,

Let's try the factorization method if \(c=c(t)\):
\[\left( \partial_{t}+c(t)\partial_{x}\right) \left(\partial_{t}-c(t)\partial_{x}\right)u=0\]
Set \(\xi=x+c(t)t\) and \(\eta=x-c(t)t\).

By the chain rule,

\[\frac{\partial}{\partial x} = \frac{\partial}{\partial \xi} \frac{\partial \xi}{\partial x}+\frac{\partial}{\partial\eta}\frac{\partial\eta}{\partial x}=
\partial_{\xi}+\partial_{\eta}\]

\[\frac{\partial}{\partial t} = \frac{\partial}{\partial \xi} \frac{\partial \xi}{\partial t}+\frac{\partial }{\partial\eta}\frac{\partial\eta}{\partial t}=\partial_{\xi}(c't+c)-\partial_{\eta}(c't+c)
\]

When we plug these back into
\[\left( \partial_{t}+c\partial_{x}\right) \left(\partial_{t}-c\partial_{x}\right)u=0\]
we get

\[\left( \partial_{\xi}(c't+c)-\partial_{\eta}(c't+c)+c\partial_{\xi}+c\partial_{\eta} \right)
\left(\partial_{\xi}(c't+c)-\partial_{\eta}(c't+c)-c\partial_{\xi}-c\partial_{\eta}\right)u=0\]
Where again we cannot cancel terms to simplify our equation as we could
with constant \(c\).

\hypertarget{problem-7-the-poisson-darboux-equation}{%
\subsection{Problem 7 (The Poisson-Darboux
Equation)}\label{problem-7-the-poisson-darboux-equation}}

Solve the initial value problem

\begin{equation} 
\begin{cases}
  u_{tt}-u_{xx}- \frac{2}{x}u_x=0 & -\infty <x<\infty, t>0 \\
  u(0,x)=0 & -\infty <x<\infty \\
  u_t(0,x)=g(x) & -\infty <x< \infty \\
\end{cases}
\end{equation}

where \(g(x)=g(-x)\) is an even function. Hint: set \(w=xu\).

Using the results from when we set \(w=ru\) in Problem 4,
\[u_t=\frac{w_t}{x}, u_{tt}=\frac{w_{tt}}{x} \]
\[u_x= \frac{w_x}{x}- \frac{w}{x^2} \]

\[u_{xx}=  \frac{w_{xx}}{x}-\frac{2w_x}{x^2}+\frac{2w}{x^3}\] So
\(u_{tt}-u_{xx}- \frac{2}{x}u_x=0\) becomes

\[\frac{w_{tt}}{x} -\frac{w_{xx}}{x}+\frac{2w_x}{x^2}- \frac{2w}{x^3}- \frac{2}{x} \left(\frac{w_x}{x}- \frac{w}{x^2}\right)=0 \]
Which simplifies to \(w_{tt}-w_{xx}=0\), and so \(w\) solves the wave
equation when \(-\infty <x<\infty, t>0\).

Now observe that \(w(0,x)=xu(0,x)=0\) and \(w_t(0,x)=xu_t(0,x)=xg(x)\).
Hence, \(w\) solves the initial value problem:

\begin{equation} \label{eqn:prob7}
    \begin{cases}
      w_{tt}-w_{xx} & -\infty <x<\infty, t>0 \\
      w(0,x)=0 & -\infty <x<\infty, t>0 \\
      w_t(0,x) = xg(x) & -\infty <x<\infty, t>0
    \end{cases}       
\end{equation}

whre \(g(x)\) is an even function.

By d'Alembert's formula, the solution to (\ref{eqn:prob7}) is given by:
\[w(t,x)=\frac{1}{2} \int_{x-t}^{x+t}{sg(s)ds}\] Hence since
\(w_t=xu_t\), \begin{equation} \label{eqn:prob7a}
w(t,x)=\frac{1}{2x} \int_{x-t}^{x+t}{sg(s)ds,}
\end{equation} assuming \(x \neq 0\).

To handle the case when \(x=0\), assume that \(u\) is continuous in
\(x\), so that \(u(t,0)=\lim_{x\to 0} u(t,x)\). Define

\[I(y)=\int_{0}^{y}sg(s)\;ds. \]

This allows us to rewrite (\ref{eqn:prob7a}) as

\[u(t,x)=\frac{I(x+t)-I(x-t)}{2x}= \frac{I(x+t)-I(t-x)}{2x},\] Noting
that the evenness of \(g(s)\) means that \(sg(s)\) is odd, and so
\(I(y)\) is even (one can check by substitution that \(I(y)=I(-y)\)).
Hence, letting \(x \rightarrow 0\), we see that

\begin{equation}\label{eqn:prob7b}
u(t,0)= \lim_{x\to 0} \frac{I(t+x)-I(t-x)}{2x}.
\end{equation}

But this is just the form of a derivative, namely the symmetric
derivative of \(I\) at \(t\). By the Fundamental Theorem of Calculus,
assuming \(g\) is continuous, we know the limit (\ref{eqn:prob7b})
exists and is necessarily equal to \(I'(t)=tg(t)\).

Thus \(u(t,x)\) is given for any \(t \geq 0, x \in \mathbb{R}\) by

\begin{equation} \label{eqn:prob7c} 
u(t,x)=
\begin{cases}
      \frac{1}{2x} \int_{x-t}^{x+t}{sg(s)ds,} & x \neq 0, \\
      tg(t), & x=0 \\
    \end{cases} 
\end{equation}

\hypertarget{problem-8}{%
\subsection{Problem 8}\label{problem-8}}

Solve the following characteristic initial value problem:
\begin{equation} 
\begin{cases}
  y^3u_{xx}-yu_{yy}+u_y=0 & 0<x<4, \quad \lvert y \rvert \leq 2 \sqrt{2} \\
  u(x,y)= f(x) & x+ \frac{y^2}{2}=4 \text{ for } 2 \leq x \leq 4 \\
  u(x,y)=g(x) & x- \frac{y^2}{2}=0 \text{ for } 0 \leq x \leq 2 \\
\end{cases}
\end{equation}

where \(f(2)=g(2)\). Hint: Use the transformation
\(\eta=x-\frac{y^2}{2}\) and \(\xi=x+\frac{y^2}{2}\) and express the PDE
in the coordinates \((\xi, \eta)\).

Set \(\eta=x-\frac{y^2}{2}\) and \(\xi=x+\frac{y^2}{2}\). Then

\[\frac{\partial u}{\partial x} = \frac{\partial u}{\partial \xi} \frac{\partial \xi}{\partial x}+\frac{\partial u}{\partial\eta}\frac{\partial\eta}{\partial x} = \frac{\partial u}{\partial \xi}+\frac{\partial u}{\partial\eta}\]

\[\frac{\partial^2 u}{\partial x^2} = \frac{\partial }{\partial x} \left(\frac{\partial u}{\partial \xi}\right) +\frac{\partial }{\partial x} \left(\frac{\partial u}{\partial \eta}\right)\]

\[=\frac{\partial^2 u}{\partial \xi^2} \frac{\partial \xi}{\partial x}+
\frac{\partial^2 u}{\partial\eta \partial \xi}\frac{\partial \xi}{\partial x} +
\frac{\partial^2 u}{\partial\eta \partial \xi}\frac{\partial\eta}{\partial x}+
\frac{\partial^2 u}{\partial \eta^2}\frac{\partial \eta}{\partial x}  \]

Which we can simplify to

\[u_{xx}= u_{\xi \xi}+2 u_{\xi \eta}+u_{\eta \eta}\]

Now we do the same for \(y\):

\[\frac{\partial u}{\partial y} = \frac{\partial u}{\partial \xi} \frac{\partial \xi}{\partial y}+\frac{\partial u}{\partial\eta}\frac{\partial\eta}{\partial y} = y\frac{\partial u}{\partial \xi}-y\frac{\partial u}{\partial\eta}\]

But since \(\eta=x-\frac{y^2}{2}\) and \(\xi=x+\frac{y^2}{2}\), we can
rewrite \(y\) as \(\sqrt{\xi-\eta}\).

\[\frac{\partial u}{\partial y} = \sqrt{\xi-\eta} \left(\frac{\partial u}{\partial \xi}-\frac{\partial u}{\partial\eta}\right)\]

Now using the product rule,
\[\frac{\partial^2 u}{\partial y^2} = \sqrt{\xi-\eta}\; \frac{\partial}{\partial y}\left(\frac{\partial u}{\partial \xi}-\frac{\partial u}{\partial\eta}\right)+\left(\frac{\partial u}{\partial \xi}-\frac{\partial u}{\partial\eta}\right)\]

Where the second term has been simplified because
\(\frac{\partial}{\partial y}\) of \(\sqrt{\xi-\eta}\) is simply
\(\frac{\partial}{\partial y}y=1\).

Now we use the chain rule again for the first term:

Now using the product rule,
\[\frac{\partial^2 u}{\partial y^2} = \sqrt{\xi-\eta}\left(\frac{\partial^2 u}{\partial \xi^2} \frac{\partial \xi}{\partial y}+
\frac{\partial^2 u}{\partial\eta \partial \xi}\frac{\partial\eta}{\partial y} -
\frac{\partial^2 u}{\partial \eta^2}\frac{\partial \eta}{\partial y}-
\frac{\partial^2 u}{\partial \eta \partial \xi}\frac{\partial \xi}{\partial y}\right)
+
\left(\frac{\partial u}{\partial \xi}-\frac{\partial u}{\partial\eta}\right)\]

Which simplifies to

\[u_{yy}= \left(\xi-\eta \right)\left(u_{\xi \xi}+u_{\eta \eta} -2 u_{\xi \eta}\right) +u_\xi-u_\eta\]

This means \(y^3u_{xx}-yu_{yy}+u_y=0\) becomes

\[\left(\xi-\eta \right)^{\frac{3}{2}} \left(u_{\xi \xi}+2 u_{\xi \eta}+u_{\eta \eta}\right) - \left(\xi-\eta \right)^{\frac{1}{2}} \left[\left(\xi-\eta \right)\left(u_{\xi \xi}-2 u_{\xi \eta}+u_{\eta \eta} \right) +u_\xi-u_\eta \right]+\left(\xi-\eta \right)^{\frac{1}{2}} \left(u_\xi-u_\eta\right)=0  \]

\[\left(\xi-\eta \right)^{\frac{3}{2}} \left(u_{\xi \xi}+2 u_{\xi \eta}+
u_{\eta \eta}\right) - \left(\xi-\eta \right)^{\frac{3}{2}} \left(u_{\xi \xi}-2 u_{\xi \eta}+
u_{\eta \eta}\right)-
\left(\xi-\eta \right)^{\frac{1}{2}} \left(u_\xi-u_\eta\right)+
\left(\xi-\eta \right)^{\frac{1}{2}} \left(u_\xi-u_\eta\right)=0  \]

\[\left(\xi-\eta \right)^{\frac{3}{2}} \left(4\, u_{\xi \eta}\right)=0\]

But by integrating with respect to both variables, we recognize
\(u_{\xi \eta}=0\) as having the solution

\[u(\xi, \eta)=h_1(\xi)+h_2(\eta),\]

or in our original coordinates,

\[u(x,y)=h_1\left(x+\frac{y^2}{2}\right)+h_2\left(x-\frac{y^2}{2}\right) \]

When \(x=2\),

\(f(x)=h_1(4)+h_2(2x-4)\)

\(z=2x-4, \quad x=\frac{z}{2} + 2\)

\(h_2(z)=f\left(\frac{z}{2}+2\right)-h_1(4)\)

and \(g(x)=h_1(2x)+h_2(0)\)

\(h_1(z)=g\left(\frac{z}{2}\right)-h_2(0)\)

\(h_1+h_2=g\left(\frac{z}{2}\right)-h_2(0)+f(\frac{z}{2} + 2)-h_1(4)\)

But \(h_2(0)+h_1(4)\) is just \(g(x)\) when \(x=2\) (or equivalently,
\(f(x)\) when \(x=2\)),

so after replacing \(z\) by \(x+\frac{y^2}{2}\) and \(x-\frac{y^2}{2}\),
we get

\begin{tcolorbox}[colback=white, title=Solution]
$$u(x,y)=g\left(\frac{x}{2}+\frac{y^2}{4}\right)+f\left(\frac{x}{2}+\frac{y^2}{4}+2\right)+f(2)$$
\end{tcolorbox}

\end{document}
